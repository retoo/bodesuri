\documentclass[12pt,halfparskip]{scrartcl}

\newcommand{\dokumenttitel}{Code}
\usepackage{../bodesuri}
\usepackage{multicol}


\begin{document}

\title{\dokumenttitel}
\titlehead{
	\centering
	\includegraphics[width=0.5 \textwidth, clip, trim = 0 7cm 0 0]{design/externes_design/bodesuri_plakat}
	\vspace{2cm}
}
\author{Danilo~Couto, Philippe~Eberli, \\ Pascal~Hobus, Reto~Schüttel, Robin~Stocker}
\maketitle
\newpage

\pagenumbering{roman}

\tableofcontents
\thispagestyle{plain}
\newpage

\pagenumbering{arabic}

\markright{Bodesuri -- \dokumenttitel}


\section{Coderichtlinien}\label{sub:coderichtlinien} % (fold)

Grundsätzlich folgen wir den "<Sun Coding Quidelines">. Zusätzlich gelten die folgenden Regeln:

\begin{itemize}
  \item Öffnende geschwungene Klammer (\texttt{\{}) auf gleicher Zeile
  \item Einrücken mit Tabulatoren, Formatieren mit Leerzeichen
  \item 1 Tabulator hat die Breite 4
  \item Maximal 80 Zeichen pro Zeile
  \item Sprache: Deutsch mit vernünftigen Ausnahmen (get und nicht gib...)
  \item Mit Javadoc kommentieren wo sinnvoll (z. B. keine Getter/Setter kommentieren)
  \item Paketnamen klein schreiben
  \item Copyright-Header\footnote{Die Möglichkeit einer Veröffentlichung des Projekts unter einer FOSS-Lizenz wird nach Abschluss des Projektes geprüft werden.} in allen Dateien:
    \begin{verbatim}
/* Copyright (c) 2007 Danilo Couto, Philippe Eberli,
 *                    Pascal Hobus, Reto Schüttel, Robin Stocker
 */
    \end{verbatim}
\end{itemize}


\paragraph{Beispielcode} (\texttt{----->} stellt einen Tabulator dar)

\begin{verbatim}
--->public void funktion(int erstens, int zweitens) {
--->--->if (erstens == zweitens) {
--->--->--->funktionZwei("Gleich");
--->--->} else {
--->--->--->funktionMitVielenArgumenten("Ungleich", erstens, zweitens,
--->--->--->                            undNochEinArgument);
--->--->}
--->}
\end{verbatim}

\section{Reviews} % (fold)
\label{sec:reviews}
Es wurden im Rahmen des Projektes verschiedene Code-Reviews durchgeführt. Sie dienten einerseits einer Qualitätskontrolle und andererseits fand dank ihnen auch ein verstärkter Know-How Transfer statt.

Ein Review dauerte in der Regel eine Stunde und wurde in wechselnden Paaren durchgeführt. Die dabei entstandenen Erkenntnissen und Aufgaben wurden im Wiki dokumentiert. Nachfolgend eine kurze Zusammenfassung der einzelnen Reviews

\subsubsection{Review Netzwerk - Pascal \& Reto}
\label{ssub:review_netzwerk_pascal_amp_reto}

\begin{itemize}
	\item Kleine Verbesserungen in den Netzwerk Klassen
	\item Verhalten bei Timeouts und Verbindungsfehlern diskutiert
	\item Weiteres Vorgehen für die Synchronisation zwischen Client und Server besprochen
\end{itemize}

\htmladdnormallink{http://bodesuri.speedpc.ch/trac/wiki/Review\%202007-05-10\%20Pascal/Reto}{http://bodesuri.speedpc.ch/trac/wiki/Review\%202007-05-10\%20Pascal/Reto}

\subsubsection{Problem Domain - Philippe \& Robin}
\label{ssub:problem_domain_philippe_amp_robin}

\begin{itemize}
	\item Kleine Refactorings (u.a. WegFeld umbenannt nacht NormalesFeld)
	\item Serialisierung vereinfacht
\end{itemize}

\htmladdnormallink{http://bodesuri.speedpc.ch/trac/wiki/Review\%202007-06-25\%20Philippe/Robin}{http://bodesuri.speedpc.ch/trac/wiki/Review\%202007-06-25\%20Philippe/Robin}

\subsubsection{GUI / Automaten Kommunikation - Danilo \& Pascal}
\label{ssub:gui_automaten_kommunikation_danilo_amp_pascal}

\begin{itemize}
	\item Kommunikation zwischen GUI und Automaten geplant
	\item Controller angeschaut und an einigen Stellen verbessert
\end{itemize}

\htmladdnormallink{http://bodesuri.speedpc.ch/trac/wiki/Review\%202007-06-11\%20Pascal/Danilo}{http://bodesuri.speedpc.ch/trac/wiki/Review\%202007-06-11\%20Pascal/Danilo}

\subsubsection{Automat - Reto \& Robin}
\label{ssub:automat_reto_amp_robin}

\begin{itemize}
	\item Synchronisation-Logik vereinfacht
	\item Neue Klasse Weg geplant die vom Regelsystem wie auch für die Wegmarkierung im UI verwendet werden kann
	\item Verschiedene kleinere Refactorings
\end{itemize}

\htmladdnormallink{http://bodesuri.speedpc.ch/trac/wiki/Review\%202007-06-11\%20Reto/Robin}{http://bodesuri.speedpc.ch/trac/wiki/Review\%202007-06-11\%20Reto/Robin}

\subsubsection{GUI - Danilo \& Philippe}
\label{ssub:gui_danilo_amp_philippe}

\begin{itemize}
	\item Lobby-View vereinfacht
	\item Setzen des Swing Look and Feels an einer zentralen Stelle
\end{itemize}

\htmladdnormallink{http://bodesuri.speedpc.ch/trac/wiki/Review\%202007-06-22\%20Danilo/Philippe}{http://bodesuri.speedpc.ch/trac/wiki/Review\%202007-06-22\%20Danilo/Philippe}

\subsubsection{User-Feeling - Philippe \& Robin}
\label{ssub:user_feeling_philippe_amp_robin}

\begin{itemize}
	\item Kleinere Bugs gefunden und behoben (z.B. falsche Reihenfolge)
	\item GUI-Handling an verschiedenen durch kleine Massnahmen verbessert
	\item Verschiedene Meldungen und Fehlermeldungen verbessert
\end{itemize}

\htmladdnormallink{http://bodesuri.speedpc.ch/trac/wiki/Review\%202007-05-11\%20Philippe/Robin}{http://bodesuri.speedpc.ch/trac/wiki/Review\%202007-05-11\%20Philippe/Robin}

\end{document}
