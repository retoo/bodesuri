\documentclass[a4paper,12pt,halfparskip,DIV14]{scrartcl}

\newcommand{\dokumenttitel}{Tests}
\usepackage{../bodesuri}
\usepackage{multicol}


\begin{document}

\title{\dokumenttitel}
\titlehead{
	\centering
	\includegraphics[width=0.5 \textwidth, clip, trim = 0 7cm 0 0]{design/externes_design/bodesuri_plakat}
	\vspace{2cm}
}
\author{Danilo~Couto, Philippe~Eberli, \\ Pascal~Hobus, Reto~Schüttel, Robin~Stocker}
\maketitle
\newpage

\pagenumbering{roman}

\tableofcontents
\thispagestyle{plain}
\newpage

\pagenumbering{arabic}

\markright{Bodesuri -- \dokumenttitel}

\section{Voraussetzungen} % (fold)
\label{sec:voraussetzungen}
Bedingungen die für einen Test vorausgesetzt werden
% section voraussetzungen (end)

\section{Vorbereitungen} % (fold)
\label{sec:vorbereitungen}
Was muss vor einem Test gemacht werden. Spezielle Einstellungen
% section vorbereitungen (end)

\section{Systemtest} % (fold)
\label{sec:systemtest}

\subsection{Tests anhand der Use-Cases}\label{sub:tests_anhand_der_use_cases} % (fold)

\subsubsection{UC1: Spiel erstellen}\label{sub:uc1_spiel_erstellen} % (fold)
\begin {tabular}{r | p{3cm} | p{8cm} | l}
\hline \hline
\textbf{Use-Case} & \textbf{Beschreibung} & \textbf{Erwartetes Verhalten} & \textbf{Ergebnis} \\
\hline
1, 2, 3 & Server-Start \newline erfolgreich & Der Server ist gestartet, funktioniert und ist betriebsbereit. & [Ok|Err] \\
 \cline{3-4} & & An den Benutzer wurde eine Bestätigung ausgegeben, dass der Server läuft. & \\
\hline
2a & Server-Start fehlgeschlagen & Der Server ist nicht gestartet und befindet sich in einem wohldefinierten Zustand. & [Ok|Err] \\
 \cline{3-4} & & An den Benutzer wurde eine Fehlermeldung ausgegeben, mit sprechenden Erklärungen und allfälligen Hinweisen die zur Behebung des Problems beitragen. & \\
\hline
1, 2 & Mehrere Server-Instanzen & FIXME: Es darf auf einem Rechner nur eine Server-Instanz laufen. (Evtl. Ports deklarierbar --> dann mehrere). & [Ok|Err] \\
\hline
\end{tabular}
% subsubsection uc1_spiel_erstellen (end)

\subsubsection{UC2: Spiel beitreten}\label{sub:uc2_spiel_beitreten} % (fold)
\begin {tabular}{r | p{3cm} | p{8cm} | l}
\hline \hline
\textbf{Use-Case} & \textbf{Beschreibung} & \textbf{Erwartetes Verhalten} & \textbf{Ergebnis} \\
\hline
1, 2, 3, 4 & Client verbunden & Der Client hat sich auf den Server verbunden und der Server kennt den Client. & [Ok|Err] \\
 \cline{3-4} & & Der Benutzer ist informiert worden, dass er dem Spiel erfolgreich beigetreten ist. & \\
\hline
2a & Verbindung fehlgeschlagen & Der Client und der Server befinden sich in einem wohldefinierten Zustand. & [Ok|Err] \\
 \cline{3-4} & & Der Benutzer wurde vom System über das Scheitern benachrichtigt. & \\
\hline
2b & Server besetzt & Der Client befindet sich in einem wohldefinierten Zustand. & [Ok|Err] \\
 \cline{3-4} & & Der bereits laufende Server wird nicht beeinflusst (Spiel läuft normal weiter). & \\
 \cline{3-4} & & Der Benutzer wurde darüber informiert, dass der Server bereit besetzt ist. & \\
\hline
*a & Verbindung abgebrochen & Der Client und der Server befinden sich in einem wohldefinierten Zustand und sind darüber informiert, dass der Benutzer die Verbindung abgebrochen hat. & [Ok|Err] \\
\hline
\end{tabular}
% subsubsection uc2_spiel_beitreten (end)

\subsubsection{UC3: Spiel spielen}\label{sub:uc3_spiel_spielen} % (fold)
\begin {tabular}{r | p{3cm} | p{8cm} | l}
\hline \hline
\textbf{Use-Case} & \textbf{Beschreibung} & \textbf{Erwartetes Verhalten} & \textbf{Ergebnis} \\
\hline
1 & Spielbrett initialisiert & Das Spielbrett ist initialisiert und auf allen Clients innerhalb der Zeitlimite von 3s synchronisiert. & [Ok|Err] \\
 \cline{3-4} & & Alle Clients und der Server befinden sich in einem wohldefinierten Zustand. & \\
\hline
2 & Karten erhalten & Der beteiligte Client hat die Spielkarten erhalten und visualisiert sie korrekt für den Spieler. & [Ok|Err] \\
\hline
3 & Karten für Tausch auswählen & Vom Benutzer ausgewählte Spielkarten sind visualisiert und als ausgewählt hervorgehoben. & [Ok|Err] \\
 \cline{3-4} & & Dem Spieler wird visuell mitgeteilt, dass das System bereit ist den Kartentausch abzuwickeln. & \\
\hline
4 & Karten-Tausch übermitteln & Die vom Benutzer getauschten Spielkarten sind an den Server übermittelt und auf dem Client (der die Spielkarten gesendet hat) nicht mehr vorhanden. & [Ok|Err] \\
 \cline{3-4} & & Der Client (der die Spielkarten gesendet hat) hat diejenigen Spielkarten vom Server empfangen, die der Mitspieler zum Tausch an den Server übermittelt hatte. & \\
 \cline{3-4} & & Für den Spieler werden die Spielkarten nach dem Tausch korrekt visualisiert. & \\
\hline
5 & Spielzüge visualisieren & Benutzer kann keine spieleingreifende Eingaben tätigen. & [Ok|Err] \\
 \cline{3-4} & & Spielzüge werden korrekt auf dem Client visualisiert. & \\
 \cline{3-4} & & Client ist nach jedem Spielzug zu den anderen Clients synchronisiert. & \\
\hline
6 & Spielzug auswählen & Vom Benutzer ausgewählte Spielkarte ist visualisiert und als ausgewählt hervorgehoben. & [Ok|Err] \\
 \cline{3-4} & & Spielzug ist vom System erfasst und für den Spieler visuell ersichtlich. & \\
 \cline{3-4} & & Dem Spieler wird visuell mitgeteilt, dass das System bereit ist den Spielzug durchzuführen. & \\
\hline
\end{tabular}

\newpage

\begin {tabular}{r | p{3cm} | p{8cm} | l}
\hline \hline
\textbf{Use-Case} & \textbf{Beschreibung} & \textbf{Erwartetes Verhalten} & \textbf{Ergebnis} \\
\hline
7, 8, 9 & Spielzug übermitteln & Spielzug ist vom Client validiert worden und gültig. & [Ok|Err] \\
 \cline{3-4} & & Spielzug ist an Server übermittelt worden und dort eingetroffen. & \\
 \cline{3-4} & & Spielzug ist vom Server validiert worden und gültig. & \\
 \cline{3-4} & & Spielzugübermittlung wurde dem Client vom Server bestätigt. & \\
 \cline{3-4} & & Alle Clients haben den Spielzug visualisiert und sind synchronisiert. & \\
\hline
6a & Kein Karte spielbar & Dem Server wurde mitgeteilt, dass Spieler diese Runde nicht spielen kann. & [Ok|Err] \\
 \cline{3-4} & & Alle Clients sind synchronisiert. & \\
\hline
7a & Spielzug ungültig & Benutzer wurde über ungültigen Zug benachrichtigt. & [Ok|Err] \\
 \cline{3-4} & & Benutzer weiss, dass er anderen Spielzug auswählen muss. & \\
\hline
9a a) & Spiel gewonnen & Auf allen Clients und dem Server ist das Spiel beendet. & [Ok|Err] \\
 \cline{3-4} & & Auf allen Clients wird eine Rangliste/Statistik angezeigt. & \\
 \cline{3-4} & & Gewinner (-Team) ist klar ersichtlich. & \\
\hline
9a b) & Spiel gewonnen & Der Server weiss, dass der Spieler fertig ist. & [Ok|Err] \\
 \cline{3-4} & & Spieler-Status ist auf allen Clients synchronisiert und für Spieler ersichtlich. & \\
 \cline{3-4} & & Der Spieler, der neu fertig ist, kann nun auch die Figuren des Mitpsielers bedienen (wenn er an der Reihe ist). & \\
\hline
*a) & Benutzer verlässt Spiel & Spiel wurde auf allen Clients und dem Server beendet. & [Ok|Err] \\
 \cline{3-4} & & Auf allen Clients wird eine Rangliste/Statistik angezeigt. & \\
 \cline{3-4} & & Gewinner (-Team) ist klar ersichtlich. & \\
\hline
*b) & Server nicht erreichbar & Benutzer wurden vom System benachrichtigt. & [Ok|Err] \\
 \cline{3-4} & & Auf allen Clients wird eine Fehlermeldung angezeigt. & \\
\hline
\end{tabular}
% subsubsection uc3_spiel_spielen (end)

% subsection tests_anhand_der_use_cases (end)

\subsection{Tests anhand der nichtfunktionalen Anforderungen}\label{sub:tests_anhand_der_nichtfunktionalen_anforderungen} % (fold)
\begin {tabular}{l | p{12cm} | l}
\hline \hline
\textbf{Nr.} & \textbf{Beschreibung} & \textbf{Eingehalten} \\
\hline
NF1 & Jeder Spielzug ist in maxmimal drei Sekunden über das Netzwerk synchronisiert. & [Ok|Err] \\
\hline
NF2 & Spielzustand ist nach jedem Spielzug in allen Clients und auf dem Server konsistent. & [Ok|Err] \\
\hline
NF3 & Inkonsistenz führt im Server und auf den Clients zu kontrolliertem Spielabbruch. & [Ok|Err] \\
\hline
NF4 & Es können nicht mehrere Spiele pro Server erstellt werden. & [Ok|Err] \\
\hline
NF5 & Ein Spiel kann nur mit der genauen Anzahl von vier Spielern gestartet werden. & [Ok|Err] \\
\hline
NF6 & Das GUI ist (bis auf die Server-Konfiguration und Verbindungsaufnahme) nur mit der Maus bedienbar. & [Ok|Err] \\
\hline
NF7 & Der Server sowie der Client ist uneingeschränkt auf den Plattformen Mac OS X 10.4, Windows XP SP2 und GNU/Linus spielbar. & [Ok|Err] \\
\hline
NF8 & Weder für den Server noch für den Client ist eine Installation oder Konfiguration (im Sinne einer Installation) notwendig. & [Ok|Err] \\
\hline
NF9 & Ausser des Spielernamens und der Serverkonfiguration müssen keine Konfigurationen eingegeben werden. & [Ok|Err] \\
\hline
\end{tabular}
% subsection tests_anhand_der_nichtfunktionalen_anforderungen (end)


% section systemtest (end)

\section{Verbesserungsmöglichkeiten} % (fold)
\label{sec:verbesserungsmöglichkeiten}
\subsection{Bekannte Einschränkungen} % (fold)
\label{sub:bekannte_einschränkungen}
bekannte Einschränkungen erwähnen und erläutern
% subsection bekannte_einschränkungen (end)
\subsection{Mögliche Detailverbesserungen} % (fold)
\label{sub:mögliche_detailverbesserungen}
daraus resultierende Verbesserungen beschreiben
% subsection mögliche_detailverbesserungen (end)
% section verbesserungsmöglichkeiten (end)

\end{document}
