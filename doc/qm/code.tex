\documentclass[a4paper,12pt,halfparskip,DIV14]{scrartcl}

\newcommand{\dokumenttitel}{Code}
\usepackage{../bodesuri}
\usepackage{multicol}


\begin{document}

\title{\dokumenttitel}
\titlehead{
	\centering
	\includegraphics[width=0.5 \textwidth, clip, trim = 0 7cm 0 0]{design/externes_design/bodesuri_plakat}
	\vspace{2cm}
}
\author{Danilo~Couto, Philippe~Eberli, \\ Pascal~Hobus, Reto~Schüttel, Robin~Stocker}
\maketitle
\newpage

\pagenumbering{roman}

\tableofcontents
\thispagestyle{plain}
\newpage

\pagenumbering{arabic}

\markright{Bodesuri -- \dokumenttitel}


\section{Coderichtlinien}\label{sub:coderichtlinien} % (fold)

Grundsätzlich folgen wir den Coderichtlinen von Java/Eclipse. Zusätzlich gelten die folgenden Regeln:

\begin{itemize}
  \item Öffnende geschwungene Klammer (\texttt{\{}) auf gleicher Zeile
  \item Einrücken mit Tabulatoren, Formatieren mit Leerzeichen
  \item 1 Tabulator hat die Breite 4
  \item Maximal 80 Zeichen pro Zeile
  \item Sprache: Deutsch mit vernünftigen Ausnahmen (get und nicht gib...)
  \item Mit Javadoc kommentieren wo sinnvoll (z. B. keine Getter/Setter kommentieren)
  \item Paketnamen klein schreiben
  \item Copyright-Header\footnote{Die Möglichkeit einer Veröffentlichung des Projekts unter einer FOSS-Lizenz wird nach Abschluss des Projektes geprüft werden.} in allen Dateien:
    \begin{verbatim}
/* Copyright (c) 2007 Danilo Couto, Philippe Eberli,
 *                    Pascal Hobus, Reto Schüttel, Robin Stocker
 */
    \end{verbatim}
\end{itemize}


\paragraph{Beispielcode} (\texttt{----->} stellt einen Tabulator dar)

\begin{verbatim}
--->public void funktion(int erstens, int zweitens) {
--->--->if (erstens == zweitens) {
--->--->--->funktionZwei("Gleich");
--->--->} else {
--->--->--->funktionMitVielenArgumenten("Ungleich", erstens, zweitens,
--->--->--->                            undNochEinArgument);
--->--->}
--->}
\end{verbatim}

\section{Reviews} % (fold)
\label{sec:reviews}

% section reviews (end)

\end{document}
