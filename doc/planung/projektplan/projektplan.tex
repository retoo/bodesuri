\documentclass[a4paper,12pt,halfparskip,DIV14]{scrartcl}

\newcommand{\dokumenttitel}{Projektplan}
\usepackage{../../bodesuri}

\begin{document}

\title{\dokumenttitel}
\titlehead{
	\centering
	\includegraphics[width=0.5 \textwidth, clip, trim = 0 7cm 0 0]{design/externes_design/bodesuri_plakat}
	\vspace{2cm}
}
\author{Danilo~Couto, Philippe~Eberli, \\ Pascal~Hobus, Reto~Schüttel, Robin~Stocker}
\maketitle
\newpage

\pagenumbering{roman}

\tableofcontents
\thispagestyle{plain}
\newpage

\pagenumbering{arabic}

\markright{Bodesuri -- \dokumenttitel}


\section{Einführung}

\subsection{Zweck}

Der Projektplan bildet die Grundlage für Bodesuri. Es beinhaltet die vollständige Planung. Weitere Bestandteile dieses Dokumentes sind die Zeitplanung und die grobe Arbeitsaufteilung auf die einzelnen Projektmitarbeiter.

\subsection{Gültigkeitsbereich}

Dieses Dokument ist die Grundlage für das ganze Projekt und hat deshalb Gültigkeit für die gesamte Projektdauer. Änderungen werden laufend ergänzt und sind in der Änderungshistorie ersichtlich.

\subsection{Definitionen und Abkürzungen}

\begin{description}
	\item[Dog] Auf den ersten Blick erinnert Dog an das "<Eile mit Weile">-Spiel. Vier eigene Spielfiguren sollen so schnell wie möglich über einen Rundkurs ins Ziel gebracht werden. Die Figuren werden jedoch nicht durch eine Würfelzahl, sondern aufgrund der Werte von Bridge-Karten möglichst vorteilhaft bewegt. Dog ist ein Partnerspiel.
\end{description}

\subsection{Referenzen}

\begin{tabular}{@{}ll@{}}
\url{http://www.netzone.ch/dog-spiel/} & Beschreibung von Dog, Spielregeln
\end{tabular}

\subsection{Übersicht}

Der folgende Abschnitt verschafft eine Übersicht des Projektes. Im Abschnitt 3 werden die Projektmitglieder vorgestellt. Als weiteres wird das Zeitmanagement erläutert und die Meilensteine festgelegt. Abschnitt 5 behandelt die möglichen Risiken und ihre Auswirkungen auf das Projekt. Darauf folgt eine Auflistung der einzelnen Arbeitspakete, die auf das Projekt präzise zusammengestellt wurden, sowie die verwendete Infrastruktur. Zum Schluss werden die Qualitätsmassnahmen gesucht und festgehalten.


\section{Projektübersicht}

Wir wollen das beliebte Brettspiel Dog als PC-Spiel realisieren, damit Dog-Fans auf der ganzen Welt gegeneinander spielen können. Die Spieler sind untereinander per Netzwerk verbunden. Pro Spieler wird je ein Client gestartet, welcher den Spielablauf visualisiert. Zusätzlich kümmert sich ein Server um die Synchronisation des Spielablaufs.

\subsection{Zweck und Ziel}

Wir möchten mit diesem Projekt unser Wissen aus den Modulen Software Engineering und Parallel- und Netzwerkprogrammierung anwenden. Ausserdem möchten wir Erfahrungen sammeln, wie ein Projekt durchgeführt wird und wie man Spiele entwickelt.

\subsection{Annahmen und Einschränkungen}

Jedem einzelnen Projektmitglied wird eine Zeitspanne von 100 Stunden zur Verfügung gestellt, um entsprechende Arbeitspakete auszuführen. Wenn am Ende noch genug Zeit vorhanden ist, werden dem Spiel weitere Features, wie zum Beispiel dem Chat, hinzugefügt. Anderenfalls werden Arbeitspakete mit der Priorität 3 nicht implementiert, um den Abgabetermin einhalten zu können.


\section{Projektorganisation}

Unser Team besteht aus fünf Mitgliedern. Alle sind gleichberechtigt.

\subsection{Organisationsstruktur}

\begin{tabular}{@{}ll@{}}
\toprule
Name                & Zuständigkeit \\
\midrule
\multicolumn{2}{@{}l}{\textbf{Teammitglieder}} \\
Edgar Danilo Couto  & Projektausführend, Grafik \\
Philippe Eberli     & Projektausführend, Dokumentation \\
Pascal Hobus        & Projektausführend, Modellierung \\
Reto Schüttel       & Projektausführend, Engine \\
Robin Stocker       & Projektausführend, Infrastruktur (SVN, Projektmanagement-Tools) \\
\midrule
\multicolumn{2}{@{}l}{\textbf{Externe Schnittstellen}} \\
Daniel Keller       & Betreuender Dozent \\
\bottomrule
\end{tabular}


\section{Management-Abläufe}

\subsection{Projekt-Kostenvoranschlag}

Das Projekt startet am 3. April 2007 und dauert bis spätestens am 6. Juli 2007. Für die Bearbeitung soll ein Richtwert von 100 Stunden pro Teammitglied angestrebt werden. Dies ergibt eine durchschnittliche Arbeitszeit von 7 bis 8 Stunden pro Woche.

Für die Einhaltung des Richtwertes kann die Stundenzahl leicht angepasst werden und/oder Features hinzugefügt oder entfernt werden.

\subsection{Projektplan}

\subsubsection{Zeitplan}

Siehe separates Dokument \texttt{Projektplan.xls}.

\subsubsection{Iterationsplanung und Meilensteine}

\begin{tabular}{@{} r r l l r @{ } l @{}}
\toprule
Start      & Ende       & Wochen & Inhalt         & \multicolumn{2}{l}{Meilenstein} \\
\midrule
3.4.2007   & 8.4.2007   & 1      & Projektantrag  & 6.4.   & Projektantrag \\
9.4.2007   & 15.4.2007  & 1      & Inception      & 17.4.  & Projektplan \\
16.4.2007  & 29.4.2007  & 2      & Elaboration 1  & 24.4.  & Requirements \\
30.4.2007  & 13.5.2007  & 2      & Elaboration 2  & 15.5.  & Prototyp \\
14.5.2007  & 27.5.2007  & 2      & Construction 1 & 29.5.  & Design und Design-Dokumentation \\
28.5.2007  & 10.6.2007  & 2      & Construction 2 \\
11.6.2007  & 24.6.2007  & 2      & Construction 3 \\
25.6.2007  & 6.7.2007   & 2      & Transition     & 6.7.   & Präsentation \\
\bottomrule
\end{tabular}

\subsubsection{Besprechungen}

Alle Beteiligten treffen sich jeweils Dienstag Nachmittags zur Teambesprechung. Wichtige Entscheidungen werden soweit möglich nur dort getroffen.\\
Für wichtige Mitteilungen wird eine Mailingliste eingerichtet. Protokolle, Traktandenlisten und weitere wichtige Dokumenten werden über ein Wiki verwaltet.

\subsubsection{Releases}

\begin{tabular}{@{}llll@{}}
\toprule
Release   & Datum       & Beschreibung\\
\midrule
Prototyp  & 1.5.2007    & "<Proof of concept"> (alle Features der Priorität 1)\\
Alpha     & 29.5.2007   & Rudimentär spielbare Version t\\
Beta      & 12.6.2007   & "<Feature complete"> (alle Features der Priorität 2)\\
Final     & 26.6.2007   & Finale Version \\
\bottomrule
\end{tabular}

\begin{tabular}{ll}
\toprule
Release 	&  Enthaltene Arbeitspakete \\
\midrule
Prototyp	&	Programmgerüst \\
					& Karten \\
					& Zugtyp Simpler Zug \\
					& Zugtyp Rückwärtszug \\
 					& Feldtyp Normal \\
					& Zugsystem \\
					& Spielzustand-Synchronisation \\
					& CLI (Text Ausgabe auf Konsole) \\
					& Externes Design erstellen / umsetzen\\
					& Zugerfassung (Simpler Zug) \\
					& Zugerfassung (Rückwärtszug) \\
					& Zusammenführen Client \\
					& Zusammenführen Regelsystem / GUI \\
					& Kartenstapel \\
					& Zusammenführen Server\\
\midrule
Alpha			& Zugtyp Startzug (Ass, König) \\
					& Zugtyp Aufteilbarer Zug (7) \\
					& Feldtyp Lager \\
					& Feldtyp Himmel \\
					& Feldtyp Bänkli \\
					& Rundensystem \\
					& Karten \\
					& Spielbrett grafisches Design \\
					& Spielbrett Einbindung (Koordinaten, etc.) \\
					& Zugvisualisierung \\
					& Zugerfassung (Startzug) \\
					& Zugerfassung (Aufteilbarer Zug) \\
\midrule
Beta			& Umwandlung Joker in beliebige Karte \\
					& Zugtyp Tausch (Bauer) \\
					& Steuerung der Figuren des Partners \\
					& Austausch von Karten \\
					& Zugerfassung (Tausch) \\
					& Einstiegsview (Lobby) \\
					& Joker verwenden\\
\midrule
Final			& \emph{Keine Arbeitspakete (Zeit um Bugs zu beheben.)}\\
\bottomrule

\end{tabular}

\section{Risiko-Management}

\begin{tabular}{@{} p{2.7cm} p{3.6cm} p{4cm} l l @{}}
\toprule
\textbf{Risiko} &
\textbf{Auswirkung} &
\textbf{Massnahmen} &
\textbf{Schaden\footnotemark (h)} &
\textbf{Priorität} \\
\midrule
Ausfall eines Teammitglieds & Verzögerungen & Zeitreserven einplanen  & $50 * 0.1 = 5$ & Mittel \\
\midrule
Datenverlust & Verlust von Teilen der Arbeit & Backup auf SVN & $100 * 0.05 = 5$ & Niedrig \\
\midrule
Fehlerhafte Zeitplanung & Nichteinhalten der Zeitvorgaben & Reserven einplanen & $20 * 0.4 = 8$ & Mittel \\
\midrule
Probleme mit Technologien (z. B. Java 2D \&  Netzwerk) & Verzögerungen durch Einarbeitungsaufwand oder Technologiewechsel & Prototyp erstellen & $40 * 0.2 = 8$ & Mittel \\
\bottomrule
\end{tabular}
\footnotetext{Berechnung des Schadens: Maximaler Schaden * Eintrittswahrscheinlichkeit = Gewichteter Schaden}


\section{Arbeitspakete}

Siehe separates Dokument \texttt{Projektplan.xls} und \url{http://bodesuri.speedpc.ch/dotproject/}

\section{Infrastruktur}

\paragraph{Räume}
\begin{itemize}
	\item Räume an der HSR
	\item Private Räume
\end{itemize}

\paragraph{Hardware}
\begin{itemize}
	\item SVN-Server
	\item Private Computer
\end{itemize}

\paragraph{Software}
\begin{itemize}
	\item Eclipse
	\item Java SDK
	\item Java 2D
	\item \LaTeX{}, Microsoft Excel
	\item Apache Ant
	\item dotProject (Zeit- \& Taskmanagement)
	\item trac (Wiki, BugTracker)
\end{itemize}


\section{Qualitätsmassnahmen}

Um die Qualität der Applikation zu gewährleisten werden folgende Massnahmen ergriffen:
\begin{itemize}
	\item \textbf{Unit-Tests}: Fortlaufende Tests der Funktionalität mittels Unit-Tests (JUnit). Es werden XX\% Codeabdeckung angestrebt.
	\item \textbf{Interne Tests}: Regelmässige, auf den Usecases basierende Tests mittels Testprotokollen
	\item \textbf{Externe Tests}: Usability-Tests über externe Tester (potenzielle Benutzer)
	\item \textbf{Bug-Tracking}: Fortlaufendes „Tracken“ von Bugs in einer Bug-Verwaltungs-Software
	\item \textbf{Continuous Integration}: Mittels Cruise-Control wird der Zustand der Applikation automatisch nach jeder Änderung getestet
	\item \textbf{Revision Control}: Der komplette Quellcode der Applikation befindet sich in einem zentralen Subversion-Repository und steht somit unter Versionskontrolle
	\item \textbf{Regelmässige Reviews}: In diesen werden die Coderichtlinien kontrolliert und geprüft, ob die Anwendung den Anforderungsspezifikationenen entspricht. 
\end{itemize}

\subsection{Coderichtlinien}\label{sub:coderichtlinien} % (fold)

Grundsätzlich folgen wir den Coderichtlinen von Java/Eclipse. Zusätzlich gelten die folgenden Regeln:

\begin{itemize}
  \item Öffnende geschwungene Klammer (\texttt{\{}) auf gleicher Zeile
  \item Einrücken mit Tabulatoren, Formatieren mit Leerzeichen
  \item 1 Tabulator hat die Breite 4
  \item Maximal 80 Zeichen pro Zeile
  \item Sprache: Deutsch mit vernünftigen Ausnahmen (get und nicht gib...)
  \item Mit Javadoc kommentieren wo sinnvoll (z. B. keine Getter/Setter kommentieren)
  \item Paketnamen klein schreiben
  \item Copyright-Header\footnote{Die Möglichkeit einer Veröffentlichung des Projekts unter einer FOSS-Lizenz wird nach Abschluss des Projektes geprüft werden.} in allen Dateien:
    \begin{verbatim}
/* Copyright (c) 2007 Danilo Couto, Philippe Eberli,
 *                    Pascal Hobus, Reto Schüttel, Robin Stocker
 */
    \end{verbatim}
\end{itemize}


\paragraph{Beispielcode} (\texttt{----->} stellt einen Tabulator dar)

\begin{verbatim}
--->public void funktion(int erstens, int zweitens) {
--->--->if (erstens == zweitens) {
--->--->--->funktionZwei("Gleich");
--->--->} else {
--->--->--->funktionMitVielenArgumenten("Ungleich", erstens, zweitens,
--->--->--->                            undNochEinArgument);
--->--->}
--->}
\end{verbatim}



\end{document}
