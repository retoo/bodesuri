\documentclass[a4paper,12pt,halfparskip,DIV14]{scrartcl}

\newcommand{\dokumenttitel}{Analyse}
\usepackage{../bodesuri}

\begin{document}

\title{\dokumenttitel}
\titlehead{
	\centering
	\includegraphics[width=0.5 \textwidth, clip, trim = 0 7cm 0 0]{design/externes_design/bodesuri_plakat}
	\vspace{2cm}
}
\author{Danilo~Couto, Philippe~Eberli, \\ Pascal~Hobus, Reto~Schüttel, Robin~Stocker}
\maketitle
\newpage

\pagenumbering{roman}

\tableofcontents
\thispagestyle{plain}
\newpage

\pagenumbering{arabic}

\markright{Bodesuri -- \dokumenttitel}


\section{Allgemeine Beschreibung}\label{sec:allgemeine_beschreibung} % (fold)
\subsection{Benutzergruppen}\label{sec:benutzergruppen} % (fold)
\begin{itemize}
	\item \textbf{Hoster}: Startet den Server auf den sich die Spieler verbinden. (Ist immer auch ein Spieler)
	\item \textbf{Spieler}: Spielt das Spiel.
\end{itemize}
% section benutzergruppen (end)
% section allgemeine_beschreibung (end)

\section{Use Cases}

\subsection{Personas}\label{sec:personas} % (fold)
\subsubsection{Hans Müller}\label{sub:hans_müller} % (fold)
Hans Müller, 28, Familienvater. Er ist gelernter Kaufmann und arbeitet als Sachbearbeiter bei einer Bank. Er nutzt den PC regelmäsig im Geschäft (Office, SAP und bankeigene Software). Surft privat ab und zu mal im Netz, um dort Informationen einzuholen. Kennt sich aber sonst eher wenig mit PCs aus. 

Er spielt seit mehreren Jahren angefressen Dog. Zu Spitzenzeiten bis zu 3 Mal pro Woche. Er möchte Bodesuri nutzen, damit er einfacher auf die Schnelle mit seinen Kollegen zocken kann. 
% subsection hans_müller (end)

\subsubsection{Maria Lehner}\label{sub:maria_lehner} % (fold)
Maria Lehner, 18, Gymnasiastin. Ist mit dem PC aufgewachsen und hat einen eigenen in ihrem Zimmer stehen. Sie nutzt ihn vor allem um mit ihren Kollegen per MSN zu Chatten und im Internet zu surfen. Ab und zu schreibt sie auch eine Arbeit für die Schule mit Word. Maria möchte später studieren gehen, am liebsten etwas im Bereich Umwelt/Natur, da sie sich dafür interessiert. 

Dog kennt sie von einem Klassenkollegen, der sie letzte Woche zu sich nach Hause zu einem Turnier eingeladen hat. Sie hat verloren und möchte nun etwas üben.
% subsection maria_lehner (end)
% section personas (end)

\subsection{Use Cases (Brief)}\label{sec:use_cases_brief_} % (fold)
\subsubsection{UC01: Spiel erstellen}\label{sub:uc01_spiel_erstellen} % (fold)
Spieler, welcher als Server agieren möchte, startet die Serverapplikation. System bestätigt erfolgreiche Erstellung.
% subsection uc01_spiel_erstellen (end)

\subsubsection{UC02: Spiel beitreten}\label{sub:uc02_spiel_beitreten} % (fold)
Spieler gibt einen Server an. System stellte eine Verbindung her und tritt dem Spiel bei. System zeigt bereits beigetretene Spieler an und wartet bis vier Spieler dem Spiel beigetreten sind. Sind vier Spieler beigetreten wird das Spiel begonnen.
% subsection uc02_spiel_beitreten (end)

\subsubsection{UC03: Spiel spielen}\label{sub:uc03_spiel_spielen} % (fold)
Das System verteilt allen Spielern Spielkarten. Jeder Spieler tauscht mit seinem Partner eine Karte. Danach werden reihum je eine Karte gezogen und die Spielfiguren entsprechend bewegt. Wenn die Spieler alle ihre Karten gespielt haben, verteilt das System wieder neue Karten und eine neue Runde beginnt. 
% subsection uc03_spiel_spielen (end)
% section use_cases_brief_ (end)
% chapter use_cases (end)

\section{Nichtfunktionale Anforderungen}\label{cha:nichtfunktionale_anforderungen} % (fold)

\subsection{Leistungs- und Mengenanforderungen}\label{sub:leistungs_und_mengenanforderungen} % (fold)
\subsubsection{Leistungsanforderungen}\label{ssub:leistungsanforderungen} % (fold)
\begin{itemize}
	\item Damit ein Spieler im Spielfluss nicht gestört wird, soll ein Spielzug in maximal drei Sekunden über das Netzwerk synchronisiert sein.
\end{itemize}
% subsubsection leistungsanforderungen (end)
\subsubsection{Mengenanforderungen}\label{ssub:mengenanforderungen} % (fold)
\begin{itemize}
	\item Es müssen genau vier Spieler für ein Spiel vorhanden sein.
	\item Es werden zwei Gruppen à zwei Spieler gebildet.
	\item Im Spiel sind zwei Decks als Kartenstapel vorhanden.
	\item Ein Deck besteht aus 52 Karten.
\end{itemize}
% subsubsection mengenanforderungen (end)

\subsubsection{Anforderungen an Schnittstellen}\label{ssub:anforderungen_an_schnittstellen} % (fold)
\paragraph{Benutzerschnittstelle}\label{ssub:benutzerschnittstelle} % (fold)
\begin{itemize}
	\item Java2D basiertes GUI.
	\item Das GUI soll so weit als möglich nur mit der Maus bedienbar sein. Ausnahmen sind zu treffende Eingaben wie z.B. die IP-Adresse für den Verbindungsaufbau zum Server, der Spielername etc.
	\item Am Ende eines Spiels soll eine einfache Statistik und eine Rangliste angezeigt werden.
	\item Das GUI repräsentiert jeweils ein Dog-Spielbrett mit den daraufliegenden Figuren.
	\item Die Karten und der Kartenstapel werden im GUI viuell dargestellt.
\end{itemize}
% paragraph benutzerschnittstelle (end)
\paragraph{Hardware-Schnittstelle}\label{ssub:hardware_schnittstelle} % (fold)
Das Spiel wird in Java entwickelt und ist Plattform unabhängig. Es wird keine weitere Hardware benötigt.
% paragraph hardware_schnittstelle (end)
\paragraph{Software-Schnittstelle}\label{ssub:software_schnittstelle} % (fold)
\begin{itemize}
	\item Java Virtual Machine
	\item Java 1.5 SDK
	\item Java 2D
	\item Java RMI (Remote Method Invocation)
\end{itemize}
% paragraph software_schnittstelle (end)
\paragraph{Datenhaltung}\label{ssub:datenhaltung} % (fold)
Es wird keine persistente Datenhaltung benötigt. Alle Daten werden jeweils nur in einer Session benötigt und müssen deshalb nicht permanent gespeichert werden.
% paragraph datenhaltung (end)
\paragraph{Kommunikationsschnittstelle}\label{ssub:kommunikationsschnittstelle} % (fold)
\begin{itemize}
	\item Als Kommunikations-Framework wird RMI (Remote Method Invocation) verwendet.
	\item Für eine Spiel-Session wird ein Server benötigt.
	\item Jeder teilnehmende Spieler (vier insgesamt) ist ein Client.
	\item Als Netzwerkprotokoll wird TCP/IP verwendet (gegeben durch RMI).
	\item Verbindungen via Firewall und Proxies werden nicht möglich sein.
\end{itemize}
% paragraph kommunikationsschnittstelle (end)
% subsubsection anforderungen_an_schnittstellen (end)

\subsection{Anforderungen im Einzelnen}\label{sub:anforderungen_im_einzelnen} % (fold)
\subsubsection{Randbedingungen für den Entwurf}\label{ssub:randbedingungen_für_den_entwurf} % (fold)
\begin{itemize}
	\item Als Coding-Standard werden weitgehend die Sun Coding Guidelines übernommen (siehe separates Dokument).
\end{itemize}
\paragraph{Einschränkungen bzgl. Software}\label{ssub:einschränkungen_bzgl_software} % (fold)
\begin{itemize}
	\item Java 1.5
	\item Betriebssysteme
	\begin{itemize}
		\item Windows XP
		\item 	Mac OS X (ab 10.4, Tiger)
	\end{itemize}
\end{itemize}
% paragraph einschränkungen_bzgl_software (end)
\paragraph{Einschränkungen bzgl. Hardware}\label{ssub:einschränkungen_bzgl_hardware} % (fold)
Für die Verwendung des Spiels ist ein netzwerkfähiger Computer vorausgesetzt.
% paragraph einschränkungen_bzgl_hardware (end)
% subsubsection randbedingungen_für_den_entwurf (end)
% subsection anforderungen_im_einzelnen (end)

\subsubsection{Merkmale}\label{ssub:merkmale} % (fold)
\paragraph{Sicherheit}\label{ssub:sicherheit} % (fold)
Es werden keine speziellen Sicherheitsvorkehrungen im Spiel getroffen. Das Spiel macht sich aber implizit die eingebauten Sicherheitsmechanismen von Java, Java2D und RMI zu Nutzen.
% paragraph sicherheit (end)
\paragraph{Wartbarkeit}\label{ssub:wartbarkeit} % (fold)
\begin{itemize}
	\item Ein Chat System soll einfach integrierbar sein.
	\item Erweiterbarkeit für die Integration in einen  Master-Server der im Internet eine Plattform bietet, um einfach einem Spiel beizutreten.
\end{itemize}
% paragraph wartbarkeit (end)
% subsubsection merkmale (end)

\subsubsection{Andere Anforderungen}\label{ssub:andere_anforderungen} % (fold)
\paragraph{Inbetriebnahme / Installation}\label{ssub:inbetriebnahme_installation} % (fold)
\begin{itemize}
	\item Die Installation soll automatisiert sein und keine Installationsanleitung benötigen.
	\item Infos zu schwierigen Schritten (z.B. Konfiguration) sollen im Programm selst gegeben werden.
\end{itemize}
% paragraph inbetriebnahme_installation (end)
\paragraph{Konfigurierbarkeit}\label{ssub:konfigurierbarkeit} % (fold)
\begin{itemize}
	\item Am Anfang eines Spiels soll ein Spieler seinen Namen / Spitznamen eingeben können.
	\item Um einem Spiel beitreten zu können soll ein Spieler lediglich die IP-Adresse des Servers angeben müssen.
\end{itemize}
% paragraph konfigurierbarkeit (end)
% subsubsection andere_anforderungen (end)

% chapter nicht_funktionale_anforderungen (end)

\section{Domainmodell}\label{cha:domainmodell} % (fold)
\subsection{Klassenspezifikationen}\label{sub:klassenspezifikationen} % (fold)
\paragraph{Regel}\label{ssub:regel} % (fold)
Eine Regel ist (selbst im Domain Model) ein abstraktes Gebilde. Eine Regel kann geweils für Karten, Züge oder Felder gelten, jedoch nicht für mehrere dieser Kategorien gleichzeitig. Dieses spezielle Regelsystem wird im Design verfeinert.
% paragraph regel (end)
\paragraph{Feldtypen}\label{ssub:feldtypen} % (fold)
Jedem Spieler sind jeweils seine eigenen Instanzen der speziellen Feldtypen Bank, Start und Himmel zugewiesen.
% paragraph feldtypen (end)
% subsection klassenspezifikationen (end)
% chapter domainmodell (end)

\section{Systemsequenzdiagramme}\label{cha:systemsequenzdiagramme} % (fold)

% chapter systemsequenzdiagramme (end)

\section{Contracts}\label{cha:contracts} % (fold)

% chapter contracts (end)

\end{document}
