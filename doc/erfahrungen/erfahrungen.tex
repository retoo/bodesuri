\documentclass[12pt,halfparskip]{scrartcl}

\newcommand{\dokumenttitel}{Erfahrungsberichte}
\usepackage{../bodesuri}


\begin{document}

\title{\dokumenttitel}
\titlehead{
	\centering
	\includegraphics[width=0.5 \textwidth, clip, trim = 0 7cm 0 0]{design/externes_design/bodesuri_plakat}
	\vspace{2cm}
}
\author{Danilo~Couto, Philippe~Eberli, \\ Pascal~Hobus, Reto~Schüttel, Robin~Stocker}
\maketitle
\newpage

\pagenumbering{roman}

\tableofcontents
\thispagestyle{plain}
\newpage

\pagenumbering{arabic}

\markright{Bodesuri -- \dokumenttitel}


\section{Team-Erfahrungsbericht}


\section{Erfahrungsbericht von Edgar Danilo Couto}

In einem solch grossen Entwicklungs-Projekt war ich bis jetzt noch nicht involviert gewesen. Wie auch schon in anderen Projekten habe ich in diesem wieder viele neue Erfahrungen gesammelt. Die wichtigste Erkenntnis war, wie wichtig die Kommunikation zwischen den einzelnen Projektmitgliedern ist und wie gross der Overhead bei einem grossen Team sein kann. Die Kommunikation zwischen den einzelnen Projektmitgliedern haben wir mittels Sitzungen, E-Mails, Chatclient und IRC sehr gut gemeistert.

Der Anreiz am Projekt war es ein PC-Spiel zu entwickeln mit einem interaktiven GUI. Es war eine neue Herausforderung ein solches GUI zu implementieren und seit einer Ewigkeit auch ein grosser Wunsch meinerseits. Am Anfang sah es danach aus, als ob Swing nicht dafür geschaffen sei und ich hatte es mit Java 2d versucht. Die Möglichkeit von geometrischen Objekten zu gestalten ist in Java 2d hervorragend gelöst. Doch das Ansprechen der einzelnen Objekte im GUI war zu kompliziert. Wir haben uns dann wieder auf  Swing entschieden. Ich war aber noch sehr skeptisch, aber es erwies sich als eine sehr gute Entscheidung. Der Code sah um einiges einfacher aus und man konnte einfach mittels Observer die Objekte ansprechen und deren neues Aussehen mitteilen. 

Sehr grossen Spass hatte ich auch beim entwerfen der Designs im Photoshop. Ich hatte nicht gedacht, dass es Möglich sei, das komplette Design in Swing zu implementieren, so dass es dem Original Design entspricht. Und mit grossem Erstaunen ist uns auch dies Geglückt. Let’s Swing. 

Als neue Technologie im GUI habe ich mich mit XML befasst. Das XML wurde für das Layout verwendet. Jede einzelne Grafik ist mittels Koordinaten auf dem Spielbrett positioniert worden. Das GUI ist so konzipiert, dass man in einem weitern Schritt ein neues Design entwerfen kann und damit spielen, ohne jeglichen Code umzuschreiben.

Ich habe mich gegen den Schluss auch noch um die Website gekümmert, spricht mit dem Wiki. Ich hatte einige bedenken, dass man damit eine ansprechende Website erstellen kann. Jetzt sehe ich das völlig Anders. Für Dokumentationen jeglicher Art ist das Wiki als Framework die richtige Entscheidung. D barrierenfreie Design ermöglicht einem durch einfache CSS Erweiterungen eine anspruchsvolle Website zu entwerfen. In den weiteren Projekten werde ich auf jeden Fall wieder auf das Wiki zurückgreifen.

Als Dokumentation haben wir LaTeX verwendet was für mich ganz neu war. Bis jetzt hatte ich nur Word als Dokumentationswerkzeug verwendet gehabt, welches aber für eine grössere Gruppe nicht geeignet ist. Durch SVN und LaTeX konnten alle Projektmitglieder zur selben Zeit am selben Dokument arbeiten, was sehr gut funktioniert hat. Doch die Möglichkeit der Gestaltung des Dokuments ist beschränkt, was ich persönlich sehr schade finde. In den kommenden Projekten werde ich wieder mit Word die Dokumente erstellen.

Zusammenfassend kann ich dazu sagen, dass das Projekt sehr lehrreich war und trotz der grossen Projektgruppe uns alles sehr gut gelungen ist. Wir hatten eine sehr schöne und amüsante Zeit zusammen, obwohl es mit sehr viel Arbeit verbunden war. Ich möchte mich auf diesem Weg noch bei allen Projektmitgliedern und Helfern für den Einsatz herzlich bedanken.


\section{Erfahrungsbericht von Pascal Hobus}

Als wir ganz am Anfang eine Arbeit suchten, die zu fünft gut umsetzbar und auch noch interessant sein soll, war ich noch etwas skeptisch und wusste nicht so recht was denn nun genau geeignet wäre. Als wir dann auf die Idee kamen das Dog-Spiel auf den Computer zu bringen war das für mich gleich ein grosser Motivationsschub, da ich das Spiel regelmässig mit meinen Freunden zuhause spiele und sehr spannend finde. Nachdem wir das Spiel auch einige male unter uns gespielt hatten und bei allen die gleiche Motivation erkennbar war machte es auch gleich doppelt Spass, da alle immer an einem Strang zogen.


Motivation						[ok]
- Dog-Spielen					[ok]
Meine Arbeiten					[nok]
- UML Diagramme erstellen		[nok]
Dokumentation					[nok]
- LaTeX							[nok]
- Magicdraw un Structure 101	[nok]
Team							[nok]
- Teamatmosphäre				[nok]
- Zusammenfassung				[nok]

\section{Erfahrungsbericht von Philippe Eberli}

\section{Erfahrungsbericht von Reto Schüttel}

\section{Erfahrungsbericht von Robin Stocker}

\end{document}
