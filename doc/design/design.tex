\documentclass[a4paper,12pt,halfparskip,DIV14]{scrartcl}

\newcommand{\dokumenttitel}{Design}
\usepackage{../bodesuri}
\usepackage{multicol}


\begin{document}

\title{\dokumenttitel}
\titlehead{
	\centering
	\includegraphics[width=0.5 \textwidth, clip, trim = 0 7cm 0 0]{design/externes_design/bodesuri_plakat}
	\vspace{2cm}
}
\author{Danilo~Couto, Philippe~Eberli, \\ Pascal~Hobus, Reto~Schüttel, Robin~Stocker}
\maketitle
\newpage

\pagenumbering{roman}

\tableofcontents
\thispagestyle{plain}
\newpage

\pagenumbering{arabic}

\markright{Bodesuri -- \dokumenttitel}


\section{Architektur} % (fold)
\label{sec:architektur}
\subsection{Architektonische Darstellung } % (fold)
\label{sub:architektonische_darstellung_}
Beschreibt die Softwarearchitektur eines Systems und wie sie sich präsentiert.
% subsection architektonische_darstellung_ (end)

\subsection{Architektonische Ziele \& Einschränkungen} % (fold)
\label{sub:architektonische_ziele_einschränkungen}
Beschreibt die Softwareanforderungen und Objekte, welche einen Einfluss auf die Architektur haben [Bspl: Safety, Security, Privacy, Distribution, …] Beinhaltet auch eine Beschreibung von Design und Implementationsstrategie, Teamstruktur, Entwicklungstools, Zeitplan, etc…
% subsection architektonische_ziele_einschränkungen (end)

\subsection{Logische Architektur} % (fold)
\label{sub:logische_architektur}
Beschreibung der logischen Struktur des Projekts
\subsubsection{Übersicht} % (fold)
\label{ssub:Übersicht}
Beschreibung mit Text und Diagramm der Architektur. Aufteilung in Packages (zum Beispiel: 3-Layer-Architektur mit  GUI, Problem Domain und Datenhaltung)
% subsubsection Übersicht (end)

\subsubsection{Design Pakete} % (fold)
\label{ssub:design_pakete}
Für jedes definierte Package erfolgt eine Beschreibung mit Diagramm.
\paragraph{Package XZY:} % (fold)
\label{ssub:package_xzy}
\subparagraph{Beschreibung des Packages} % (fold)
\label{ssub:beschreibung_des_packages}
Beschreibung des Package. Aufgabe, etc…
% subparagraph beschreibung_des_packages (end)
\subparagraph{Diagramme} % (fold)
\label{ssub:diagramme}
Klassendiagramm
% subparagraph diagramme (end)
\subparagraph{Schnittstellen} % (fold)
\label{ssub:schnittstellen}
Beschreibung der Schnittstellen
% subparagraph schnittstellen (end)
\subparagraph{Operationen} % (fold)
\label{ssub:operationen}
Beschreibung der internen Operationen (evtl auch mit Sequendiagramm).
% subparagraph operationen (end)
% paragraph package_xzy (end)
% subsubsection design_pakete (end)
% subsection logische_architektur (end)

\subsection{Prozesse \& Threads} % (fold)
\label{sub:prozesse_threads}
Beschrieben, wie diese ablaufen, miteinander funktionieren, Daten austauschen, sich synchronisieren, etc...
% subsection prozesse_threads (end)

\subsection{Grössen \& Leistung} % (fold)
\label{sub:grössen_und_leistung}
Einschränkungen der Applikation bezüglich Speicher, Leistung, etc…. (zum Beispiel: Verwaltung unterstützt maximal 20'000 Einträge)
% subsection grössen_und_leistung (end)

% section architektur (end)

\section{Klassendiagramm} % (fold)
\label{sec:klassendiagramm}

% section klassendiagramm (end)
\end{document}
