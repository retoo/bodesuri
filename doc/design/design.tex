\documentclass[a4paper,12pt,halfparskip,DIV14]{scrartcl}

\newcommand{\dokumenttitel}{Design}
\usepackage{../bodesuri}
\usepackage{multicol}


\begin{document}

\title{\dokumenttitel}
\titlehead{
	\centering
	\includegraphics[width=0.5 \textwidth, clip, trim = 0 7cm 0 0]{design/externes_design/bodesuri_plakat}
	\vspace{2cm}
}
\author{Danilo~Couto, Philippe~Eberli, \\ Pascal~Hobus, Reto~Schüttel, Robin~Stocker}
\maketitle
\newpage

\pagenumbering{roman}

\tableofcontents
\thispagestyle{plain}
\newpage

\pagenumbering{arabic}

\markright{Bodesuri -- \dokumenttitel}


\section{Architekturübersicht} % (fold)
\label{sec:architekturübersicht}
Bodesuri ist über mehrere Computer aufgeteilt. Grundbaustein der Kommunikation ist die ZugEingabe. Diese wird vom Spieler der am Zug ist an den Server geschickt. Sie enthält den Zug, den Spieler und die gezogene Karte. Der Server verteilt diese Informationen als ZugInformation dann an alle Spieler.
\begin{figure}
	[htp] \centering 
	\includegraphics[width=0.50\textwidth]{clientServer.png} \caption{Kommunikation zwischen Client & Server}\label{fig:legende.png} 
\end{figure}

Die Validierung der gespielten Züge geschieht beim Client, der Server ist zuständig für den Spielablauf.
% section architekturübersicht (end)

\section{Systemstruktur} % (fold)
\label{Systemstruktur}

\subsection{Architekturübersicht} % (fold)
\label{sub:architekturuebersicht}
\subsection{Architektonische Darstellung} % (fold)
\label{sub:architektonische_darstellung}
Für die logische Strukturierung des Projektes wird eine vierschichtige Architektur verwendet. Dabei setzt jede Schicht direkt auf der darunterliegenden Schicht auf und nutzt deren angebotenen Dienste. Wo möglich und sinnvoll, werden solche Schnittstellen zwischen Layern mit dem Facade-Pattern gebündelt um die Kopplung zwischen den Schichten zu verringern (wenige, starke Kopplungen sind besser als viele, schwache). Dadurch lassen sich die Schnittstellen für die Schichten auch klarer definieren und sind einfacher zu nutzen.
% subsection architekturuebersicht (end)

\subsection{Architektonische Ziele \& Einschränkungen} % (fold)
\label{sub:architektonische_ziele_einschraenkungen}
\paragraph{Designentscheidungen}\label{ssub:designentscheidungen} % (fold)
Die Applikationsschicht wurde deshalb eingeführt, weil es wegen der Netzwerkfähigkeit viel Synchronisationsarbeit (zB. für Zustände) gibt. Die Zustandssynchronisation ist die erste Vorstufe der Netzwerkkommunikation. In diesem Package werden die spezifischen Nachrichten eingepackt, an die darunterliegenden Schichten weitergeleitet, um dann schlussendlich über das Netzwerk verteilt und empfangen zu werden.
% paragraph designentscheidungen (end)
\paragraph{Verwendete Entwicklungswerkzeuge}\label{ssub:verwendete_entwicklungswerkzeuge} % (fold)
Als Entwicklungswerkzeug wird MagicDraw UML verwendet. Es arbeitet Plattformunabhängig (Java), ist einfach handhabbar und hat sehr gute Code-Generierung Funktionalität. Die erzeugten Dateien lassen sich im XML Format im SVN Repository verwalten. Ein Nachteil ist, dass immer nur eine Person gleichzeitig an einem Projekt / Modul arbeiten kann, da MagicDraw UML keine SVN Integration bietet.
% paragraph verwendete_entwicklungswerkzeuge (end)
\paragraph{Teamstruktur}\label{ssub:teamstruktur} % (fold)
Die Aufteilung des Teams ist aus den Arbeitspaketen in der Excel-Zeiterfassung ersichtlich. Die Verantwortlichkeiten sind in der Analysespzifikation geregelt.
% paragraph teamstruktur (end)
% subsection architektonische_ziele_einschraenkungen (end)

\newline
\subsection{Logische Sicht} % (fold)
\label{sub:logische_schicht}

\subsubsection{Legende für die verwendeten Farben in den Diagrammen} % (fold)
\label{ssub:legende_fuer_die_verwendeten_farben_in_den_diagrammen}
\begin{figure}
	[htp] \centering 
	\includegraphics[width=1\textwidth]{legende.png} \caption{Legende der Diagrammfarben}\label{fig:legende.png} 
\end{figure}
% subsubsection legende_fuer_die_verwendeten_farben_in_den_diagrammen (end)

\subsubsection{Schichtenarchitektur} % (fold)
\label{sub:schichtenarchitektur}
\begin{figure}
	[htp] \centering 
	\includegraphics[width=1\textwidth]{Architektur.png} \caption{Architektur}\label{fig:Architektur.png} 
\end{figure}
% subsubsection schichtenarchitektur (end)

\subsubsection{Client Architektur} % (fold)
\label{sub:client_architektur}
% subsubsection client_architektur (end)

\subsubsection{Server Architektur} % (fold)
\label{sub:server_architektur}
% subsubsection server_architektur (end)

\subsubsection{Packagebeschreibung} % (fold)
\label{sub:pagebeschreibung}
% subsubsection pagebeschreibung (end)

\subsubsection{Package pd} % (fold)
\label{ssub:package_pd}
\subparagraph{Beschreibung des Packages} % (fold)
\label{ssub:beschreibung_des_packages}
In der Problem-Domain Schicht wird die gesamte Spiellogik gekapselt. Sie wird direkt von der darüberliegenden Applikationsschicht verwendet.
% subparagraph beschreibung_des_packages (end)
\subparagraph{Schnittstellen} % (fold)
\label{ssub:schnittstellen}
\begin{itemize}
	\item Package pd.zustandsynchronisation wird von applikation.zustandsynchronisation verwendet um Nachrichten vorzubereiten und an die darunterliegende Schicht zum Versenden weiterzuleiten.
	\item Package pd.zugsystem wird von applikation.zugentgegennahme verwendet, um Züge zu verarbeiten.
\end{itemize}
% subparagraph schnittstellen (end)
% subsubsection package_pd (end)

\newpage
\subsubsection{Package pd.karten} % (fold)
\label{ssub:package_pd_karten}
\subparagraph{Beschreibung des Packages} % (fold)
\label{ssub:beschreibung_des_packages}
Beinhaltet den Kartengeber, welcher die Verwaltungsklasse für gesamte Handhabung der Karten ist.
% subparagraph beschreibung_des_packages (end)
\subparagraph{Diagramme} % (fold)
\label{ssub:diagramme}
\begin{figure}
	[htp] \centering 
	\includegraphics[width=1\textwidth]{pd_kartengeber.png} \caption{Kartengeber}\label{fig:pd_kartengeber.png} 
\end{figure}
% subparagraph diagramme (end)
\subparagraph{Schnittstellen} % (fold)
\label{ssub:schnittstellen}
Beschreibung der Schnittstellen
% subparagraph schnittstellen (end)
\subparagraph{Operationen} % (fold)
\label{ssub:operationen}
Beschreibung der internen Operationen (evtl auch mit Sequendiagramm).
% subparagraph operationen (end)
% subsubsection package_pd_karten (end)

\newpage
\subsubsection{Package pd.regelsystem, pd.zugsystem} % (fold)
\label{ssub:package_pd_regelsystem}
\subparagraph{Beschreibung des Packages} % (fold)
\label{ssub:beschreibung_des_packages}
Das Regel- und das Zugsystem sind eng miteinander verbunden. Sie nehmen die Aufgaben der Problem-Domain Schicht war und stellen die Validierung der Züge und der Aktionen sicher.
% subparagraph beschreibung_des_packages (end)
\subparagraph{Diagramme} % (fold)
\label{ssub:diagramme}
\begin{figure}
	[htp] \centering 
	\includegraphics[width=1\textwidth]{pd_regelsystem.png} \caption{Regelsystem}\label{fig:pd_regelsystem.png} 
\end{figure}
% subparagraph diagramme (end)
\subparagraph{Schnittstellen} % (fold)
\label{ssub:schnittstellen}
Beschreibung der Schnittstellen
% subparagraph schnittstellen (end)
\subparagraph{Operationen} % (fold)
\label{ssub:operationen}
\begin{figure}
	[htp] \centering 
	\includegraphics[width=1\textwidth]{pd_validierung.png} \caption{Validierung von Spielzügen}\label{fig:pd_validierung.png} 
\end{figure}
% subparagraph operationen (end)
% paragraph package_pd_regelsystem (end)

\newpage
\subsubsection{Package pd.brett, pd.spieler} % (fold)
\label{ssub:package_pd_brett}
\subparagraph{Beschreibung des Packages} % (fold)
\label{ssub:beschreibung_des_packages}
Die beiden Packages sind eng miteinander verbunden und stellen die Abstraktion des Bretts, der Figuren und des Spielers in der Problem Domain dar.
% subparagraph beschreibung_des_packages (end)
\subparagraph{Diagramme} % (fold)
\label{ssub:diagramme}
\begin{figure}
	[htp] \centering 
	\includegraphics[width=1\textwidth]{pd_brett.png} \caption{Brett und Figuren}\label{fig:pd_brett.png} 
\end{figure}
% subparagraph diagramme (end)
\subparagraph{Schnittstellen} % (fold)
\label{ssub:schnittstellen}
Beschreibung der Schnittstellen
% subparagraph schnittstellen (end)
\subparagraph{Operationen} % (fold)
\label{ssub:operationen}
Beschreibung der internen Operationen (evtl auch mit Sequendiagramm).
% subparagraph operationen (end)
% paragraph package_pd_brett (end)

\newpage
\subsubsection{Package pd.zustandsynchronisation} % (fold)
\label{ssub:package_pd_zustandsynchronisation}
\subparagraph{Beschreibung des Packages} % (fold)
\label{ssub:beschreibung_des_packages}
Beschreibung des Package. Aufgabe, etc…
% subparagraph beschreibung_des_packages (end)
\subparagraph{Diagramme} % (fold)
\label{ssub:diagramme}
Klassendiagramm
% subparagraph diagramme (end)
\subparagraph{Schnittstellen} % (fold)
\label{ssub:schnittstellen}
Beschreibung der Schnittstellen
% subparagraph schnittstellen (end)
\subparagraph{Operationen} % (fold)
\label{ssub:operationen}
Beschreibung der internen Operationen (evtl auch mit Sequendiagramm).
% subparagraph operationen (end)
% paragraph package_pd_zustandsynchronisation (end)

\newpage
\subsubsection{Package dienste} % (fold)
\label{ssub:package_dienste}
\subparagraph{Beschreibung des Packages} % (fold)
\label{ssub:beschreibung_des_packages}
Die Dienstschicht stellt die grundlegende Netzwerkkommunikation dar. Er kapselt die Java Socket Schnittstelle und stellt eine abstrakte Schnittstelle der Problem-Domain Schicht zur Verfügung.
% subparagraph beschreibung_des_packages (end)
\subparagraph{Schnittstellen} % (fold)
\label{ssub:schnittstellen}
Beschreibung der Schnittstellen
% subparagraph schnittstellen (end)
% paragraph package_dienste (end)

\newpage
\subsubsection{Package dienste.serialisierung} % (fold)
\label{ssub:package_dienste_serialisierung}
\subparagraph{Beschreibung des Packages} % (fold)
\label{ssub:beschreibung_des_packages}
Dieses Package ist für die Serialisierung und Deserialisierung der Java-Objekte zuständig, welche anschliessend über das Netzwerk versendet werden.
% subparagraph beschreibung_des_packages (end)
\subparagraph{Diagramme} % (fold)
\label{ssub:diagramme}
Klassendiagramm
% subparagraph diagramme (end)
\subparagraph{Schnittstellen} % (fold)
\label{ssub:schnittstellen}
Beschreibung der Schnittstellen
% subparagraph schnittstellen (end)
\subparagraph{Operationen} % (fold)
\label{ssub:operationen}
Beschreibung der internen Operationen (evtl auch mit Sequendiagramm).
\begin{figure}
	[htp] \centering 
	\includegraphics[width=1\textwidth]{dienste_serialisierung.png} \caption{Serialisierung}\label{fig:dienste_serialisierung.png} 
\end{figure}
% subparagraph operationen (end)
% paragraph package_dienste_serialisierung (end)

\newpage
\subsubsection{Package dienste.netzwerk} % (fold)
\label{ssub:package_dienste_netzwerk}
\subparagraph{Beschreibung des Packages} % (fold)
\label{ssub:beschreibung_des_packages}
Dieses Package kapselt die Java-Socket Schnittstelle und bietet Dienste für die Netzwerkkommunikation an.
% subparagraph beschreibung_des_packages (end)
\subparagraph{Diagramme} % (fold)
\label{ssub:diagramme}
Klassendiagramm
% subparagraph diagramme (end)
\subparagraph{Schnittstellen} % (fold)
\label{ssub:schnittstellen}
Beschreibung der Schnittstellen
% subparagraph schnittstellen (end)
\subparagraph{Operationen} % (fold)
\label{ssub:operationen}
Beschreibung der internen Operationen (evtl auch mit Sequendiagramm).
\begin{figure}
	[htp] \centering 
	\includegraphics[width=1\textwidth]{dienste_partnerschaft_normal_1.png} \caption{Normale Partnerschaft - Teil 1 von 2}\label{fig:dienste_partnerschaft_normal_1.png} 
\end{figure}
\begin{figure}
	[htp] \centering 
	\includegraphics[width=1\textwidth]{dienste_partnerschaft_normal_2.png} \caption{Normale Partnerschaft - Teil 2 von 2}\label{fig:dienste_partnerschaft_normal_2.png} 
\end{figure}
\begin{figure}
	[htp] \centering 
	\includegraphics[width=1\textwidth]{dienste_partner.png} \caption{Partner Verhalten}\label{fig:dienste_partner.png} 
\end{figure}
\begin{figure}
	[htp] \centering 
	\includegraphics[width=1\textwidth]{dienste_rundenstart.png} \caption{Rundenstart}\label{fig:dienste_rundenstart.png} 
\end{figure}
\begin{figure}
	[htp] \centering 
	\includegraphics[width=0.8\textwidth]{dienste_client.png}
	\caption{Zustände des Client}\label{fig:dienste_client.png} 
\end{figure}
\begin{figure}
	[htp] \centering 
	\includegraphics[width=1\textwidth]{dienste_chat.png} \caption{Chat Subsystem}\label{fig:dienste_chat.png} 
\end{figure}
% subparagraph operationen (end)
% paragraph package_dienste_netzwerk (end)

\newpage
\subsubsection{Package ui} % (fold)
\label{ssub:package_ui}
\subparagraph{Beschreibung des Packages} % (fold)
\label{ssub:beschreibung_des_packages}
Die UI Schicht ist für die graphische Darstellung verantwortlich. Sie kommuniziert mit der Applikationsschicht top-down über direkte Assoziationen und bottom-up über Observer.
% subparagraph beschreibung_des_packages (end)
\subparagraph{Diagramme} % (fold)
\label{ssub:diagramme}
Klassendiagramm
% subparagraph diagramme (end)
\subparagraph{Schnittstellen} % (fold)
\label{ssub:schnittstellen}
Beschreibung der Schnittstellen
% subparagraph schnittstellen (end)
\subparagraph{Operationen} % (fold)
\label{ssub:operationen}
Beschreibung der internen Operationen (evtl auch mit Sequendiagramm).
% subparagraph operationen (end)
% paragraph package_ui (end)

\newpage
\subsubsection{Package ui.brett} % (fold)
\label{ssub:package_ui_brett}
\subparagraph{Beschreibung des Packages} % (fold)
\label{ssub:beschreibung_des_packages}
Beschreibung des Package. Aufgabe, etc…
% subparagraph beschreibung_des_packages (end)
\subparagraph{Diagramme} % (fold)
\label{ssub:diagramme}
Klassendiagramm
% subparagraph diagramme (end)
\subparagraph{Schnittstellen} % (fold)
\label{ssub:schnittstellen}
Beschreibung der Schnittstellen
% subparagraph schnittstellen (end)
\subparagraph{Operationen} % (fold)
\label{ssub:operationen}
Beschreibung der internen Operationen (evtl auch mit Sequendiagramm).
% subparagraph operationen (end)
% paragraph package_ui_brett (end)

\newpage
\subsubsection{Package ui.eigenschaften} % (fold)
\label{ssub:package_ui_eigenschaften}
\subparagraph{Beschreibung des Packages} % (fold)
\label{ssub:beschreibung_des_packages}
Beschreibung des Package. Aufgabe, etc…
% subparagraph beschreibung_des_packages (end)
\subparagraph{Diagramme} % (fold)
\label{ssub:diagramme}
Klassendiagramm
% subparagraph diagramme (end)
\subparagraph{Schnittstellen} % (fold)
\label{ssub:schnittstellen}
Beschreibung der Schnittstellen
% subparagraph schnittstellen (end)
\subparagraph{Operationen} % (fold)
\label{ssub:operationen}
Beschreibung der internen Operationen (evtl auch mit Sequendiagramm).
% subparagraph operationen (end)
% paragraph package_ui_eigenschaften (end)

\newpage
\subsubsection{Package ui.info} % (fold)
\label{ssub:package_ui_info}
\subparagraph{Beschreibung des Packages} % (fold)
\label{ssub:beschreibung_des_packages}
Beschreibung des Package. Aufgabe, etc…
% subparagraph beschreibung_des_packages (end)
\subparagraph{Diagramme} % (fold)
\label{ssub:diagramme}
Klassendiagramm
% subparagraph diagramme (end)
\subparagraph{Schnittstellen} % (fold)
\label{ssub:schnittstellen}
Beschreibung der Schnittstellen
% subparagraph schnittstellen (end)
\subparagraph{Operationen} % (fold)
\label{ssub:operationen}
Beschreibung der internen Operationen (evtl auch mit Sequendiagramm).
% subparagraph operationen (end)
% paragraph package_ui_info (end)

\newpage
\subsubsection{Package ui.ressourcen} % (fold)
\label{ssub:package_ui_ressourcen}
\subparagraph{Beschreibung des Packages} % (fold)
\label{ssub:beschreibung_des_packages}
Beschreibung des Package. Aufgabe, etc…
% subparagraph beschreibung_des_packages (end)
\subparagraph{Diagramme} % (fold)
\label{ssub:diagramme}
Klassendiagramm
% subparagraph diagramme (end)
\subparagraph{Schnittstellen} % (fold)
\label{ssub:schnittstellen}
Beschreibung der Schnittstellen
% subparagraph schnittstellen (end)
\subparagraph{Operationen} % (fold)
\label{ssub:operationen}
Beschreibung der internen Operationen (evtl auch mit Sequendiagramm).
% subparagraph operationen (end)
% paragraph package_ui_ressourcen (end)

\newpage
\subsubsection{Package applikation} % (fold)
\label{ssub:package_applikation}
\subparagraph{Beschreibung des Packages} % (fold)
\label{ssub:beschreibung_des_packages}
Die Applikationsschicht ist für die Zustandsynchronisation (Spielstände usw.) verantwortlich. 
% subparagraph beschreibung_des_packages (end)
\subparagraph{Diagramme} % (fold)
\label{ssub:diagramme}
Klassendiagramm
% subparagraph diagramme (end)
\subparagraph{Schnittstellen} % (fold)
\label{ssub:schnittstellen}
Beschreibung der Schnittstellen
% subparagraph schnittstellen (end)
\subparagraph{Operationen} % (fold)
\label{ssub:operationen}
Beschreibung der internen Operationen (evtl auch mit Sequendiagramm).
% subparagraph operationen (end)
% paragraph package_applikation (end)

\newpage
\subsubsection{Package applikation.zugentgegennahme} % (fold)
\label{ssub:package_applikation_zugentgegennahme}
\subparagraph{Beschreibung des Packages} % (fold)
\label{ssub:beschreibung_des_packages}
Beschreibung des Package. Aufgabe, etc…
% subparagraph beschreibung_des_packages (end)
\subparagraph{Diagramme} % (fold)
\label{ssub:diagramme}
Klassendiagramm
% subparagraph diagramme (end)
\subparagraph{Schnittstellen} % (fold)
\label{ssub:schnittstellen}
Beschreibung der Schnittstellen
% subparagraph schnittstellen (end)
\subparagraph{Operationen} % (fold)
\label{ssub:operationen}
Beschreibung der internen Operationen (evtl auch mit Sequendiagramm).
% subparagraph operationen (end)
% paragraph package_applikation_zugentgegennahme (end)

\newpage
\subsubsection{Package applikation.zustandsynchronisation} % (fold)
\label{ssub:package_app_zustandsynchronisation}
\subparagraph{Beschreibung des Packages} % (fold)
\label{ssub:beschreibung_des_packages}
Beschreibung des Package. Aufgabe, etc…
% subparagraph beschreibung_des_packages (end)
\subparagraph{Diagramme} % (fold)
\label{ssub:diagramme}
Klassendiagramm
% subparagraph diagramme (end)
\subparagraph{Schnittstellen} % (fold)
\label{ssub:schnittstellen}
Beschreibung der Schnittstellen
% subparagraph schnittstellen (end)
\subparagraph{Operationen} % (fold)
\label{ssub:operationen}
Beschreibung der internen Operationen (evtl auch mit Sequendiagramm).
% subparagraph operationen (end)
% paragraph package_app_zustandsynchronisation (end)

% subsubsection design_pakete (end)
% subsection logische_architektur (end)

\newpage
\subsection{Physikalische Schicht} % (fold)
\label{sub:physikalische_schicht}

\subsubsection{Deployment Diagramm} % (fold)
\label{sub:deployment_diagramm}

\subsubsection{Prozess \& Threads} % (fold)
\label{sub:prozess_threads}
\label{sub:prozesse_threads}
Beschrieben, wie diese ablaufen, miteinander funktionieren, Daten austauschen, sich synchronisieren, etc...
% subsection prozesse_threads (end)

\newpage
\subsection{Schnittstellen der Packages} % (fold)
\label{sub:schnittstellen_der_packages}

\newpage
\section{Spielzustände \& Nachrichten} % (fold)
\label{spielzustaende_nachrichten}

\newpage
\section{Design Pakete} % (fold)
\label{design_pakete}

\newpage
\section{Dynamische Abläufe} % (fold)
\label{dynamische_ablauefe}

\newpage
\section{Externes Design} % (fold)
\label{externes_design}

\begin{figure}[h]
	\centering
		\includegraphics[width=0.80\textwidth]{Externes Design - Verbindung.png}
		\caption{Verbindung zum Server}
	\label{fig:Externes Design - Verbindung}
\end{figure}

\begin{figure}	[htp] \centering 
	\includegraphics[width=0.8\textwidth]{Externes Design - Lobby.png} \caption{Lobby vor dem Spielbeginn}\label{fig:Externes_Design.jpg} 
\end{figure}
\begin{figure}
	[htp] \centering 
	\includegraphics[width=0.8\textwidth]{Externes Design - Spiel.png} \caption{Spielbrett}\label{fig:Externes_Design.jpg} 
\end{figure}

% subsection externes_design (end)

\newpage
\section{Eingetretene Risiken} % (fold)
\label{eingetretene_risiken}

\subsection{RMI} % (fold)
\label{sub:rmi}

Im Verlauf der ersten Tests mit RMI stellte sich schnell heraus, dass RMI weit mehr kann als für das Projekt notwendig wäre und es dem Projekt somit unnötig Komplexität hinzufügt. Andererseits stellt RMI  einige Anforderungen an die Client/Server-Struktur, die sich nur schwer mit den Projektanforderungen decken lassen. So kann RMI nur erschwert hinter Firewalls betrieben werden und eine Kommunikation über ein durch NAT\footnote{Network Adress Translation. Firewall-Feature welches Netzbereiche auf andere Netzbereiche abbildet. Wird häufig verwendet, um private Adressbereiche im Internet hinter einer einzelnen Adresse zu verstecken.} verstecktes Netzwerk ist gar nicht erst möglich. Da die Internet-Zugangslösungen in den meisten Haushalten auf Firewalls und NAT basieren, könnte das Spiel von einem grösseren Teil der potenziellen Kundschaft gar nicht gespielt werden.

Wir entschieden uns auf Grund dieser Probleme, auf eine eigene Lösung umzusteigen, welche auf den Java-Klassen Socket und Object(Input|Output)Stream basiert. Insgesamt gingen etwa 10 Stunden Arbeit für diese Umstellung verloren.
% subsection rmi (end)

\subsection{Java 2D} % (fold)
\label{java_2d}

Nachdem das CLI (Command Line Interface) erstellt war, entwarfen wir ein GUI dafür, das die Felder und Figuren darstellen sollte. Mit Java 2D ist das Zeichnen sehr einfach, da man die Koordinaten der zu zeichnenden Elemente angeben kann. Doch mussten wir leider feststellen, dass das Ansprechen eines Objektes etwas komplizierter ist. So muss zum Beispiel bei einem Mausklick das geklickte Objekt über die Koordinaten ermittelt werden. Ausserdem muss man sich um das Aktualisieren der Anzeige bei Veränderungen selber kümmern.

In einem zweiten Versuch erstellten wir das selbe Spielbrett mit Swing und wir kamen zum Schluss, dass dies die einfachere Methode ist. Die Objekte können mit einem speziellen Layout auch auf Koordinaten genau platziert werden und Klicke darauf kann man wie gewohnt mit einem MouseListener abfangen. Als Objekt kann man zum Beispiel ein Label mit einem Icon verwenden. Dies reicht für ein Brettspiel vollkommen aus.

Aus diesen Gründen haben wir uns dazu entschlossen, für das GUI als einzige Technologie Swing einzusetzen.

% subsection java_2d (end)

% section eingetretene_risiken (end)

\end{document}
